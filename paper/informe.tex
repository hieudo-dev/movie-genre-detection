\documentclass[]{article}

% Spanish accents	
\usepackage[spanish]{babel}
\usepackage[utf8]{inputenc}


%opening
\title{Detecci\'on de g\'eneros de pel\'iculas usando im\'agenes publicitarios}
\author{H JD A}

\begin{document}

\maketitle

\begin{abstract}
Se propone lograr la detecci\'on de g\'eneros de pel\'iculas usando s\'olamente im\'agenes publicitarios de estas. Una red neuronal es entrenada para obtener informaci\'on latente contenida en las im\'agenes y usando estos, detectar  los g\'eneros de una pel\'icula, dado una imagen publicitaria. Como las pel\'iculas pueden pertenecer a m\'ultiples g\'eneros, el problema se reduce a un problema de \textit{Multi-label image clasification\footnote{Clasificaci\'on de im\'agenes de varias etiquetas}}. Para agilizar la implementaci\'on de la propuesta y facilitar su comparaci\'on con otras propuestas del estado del arte, se reutiliz\'o una base de datos de im\'agenes tomado de un trabajo relacionado \cite{weitachu}. Basado en este dataset, {TODO}. En la evaluaci\'on, se muestra los resultados de la propuesta, el cual obtiene un rendimiento bastante alentador comparado con los trabajos relacionados. Para
\end{abstract}

\section{Introducci\'on}

\newpage
\bibliographystyle{plain}
\bibliography{Bibliography}

\end{document}
